\RequirePackage{etex}
\RequirePackage{easymat}
\documentclass[usenames,dvipsnames]{beamer}
\usetheme{Gelugor}
\usefonttheme[onlylarge]{structurebold}
\setbeamerfont*{frametitle}{size=\normalsize,series=\bfseries}
\setbeamertemplate{navigation symbols}{}


\usepackage{amsthm,ulsy,amsmath,amssymb,
MnSymbol,tikz,bbold}
\usepackage[all]{xy}
\usepackage[active]{srcltx}
\sloppy




\renewcommand{\le}{\leqslant}
\renewcommand{\ge}{\geqslant}

\renewcommand{\dashrightarrow}
{\text{\raisebox{0.9mm} {\
\begin{tikzpicture}[->,thick,xscale=0.56]
  \draw[dashed] (0,0)--(1,0)
;
\end{tikzpicture}}\ }}

\renewcommand{\dashleftarrow}
{\text{\raisebox{0.9mm} {\
\begin{tikzpicture}[<-,thick,xscale=0.56]
  \draw[dashed] (0,0)--(1,0)
;
\end{tikzpicture}}\ }}


\setcounter{page}{2}
\usepackage{color}

\definecolor{lavand}{RGB}{51,0,102}
\mode<presentation>
\title{Generalizations of Roth's criteria
for solvability
of matrix equations}
%\subtitle{Yet Another Grassroot Effort}
\author{Tetiana Klymchuk}
\date{\today}
\institute{\url{tetiana.klymchuk@upc.edu}\\\url{http://www.upc.edu/}}
\begin{document}
\begin{frame}[plain,t]
\titlepage
\end{frame}



\begin{frame}
\frametitle{
\includegraphics<1>[height=1.2cm]{configuracion_1.png}
\hspace{2cm}
\includegraphics<1>[height=1.2cm]{upc.png}
\hspace{0.3cm}
\includegraphics<1>[height=1.2cm]{knu.jpg}
}
\begin{center}
\bigskip

{\Large\alert{
Generalizations of Roth's criteria
for solvability \\
of matrix equations}}
\bigskip

{\large
{\it Tetiana Klymchuk}}\\
\medskip

{\large
{Universitat Polit\`{e}cnica de Catalunya}}\\
\vspace{1cm}


\textcolor{lavand}{\large
XXV CEDYA / XV CMA}\\
\medskip

\textcolor{lavand}{June 26--30, 2017}

\end{center}

\end{frame}






\section{Introduction}
\begin{frame}[t]
\frametitle{Semilinear mappings}

A mapping ${\cal A}$ from a complex
vector space $U$ to a complex vector
space $V$ is \alert{semilinear} if
\[
{\cal A}(u+u')={\cal A}u+{\cal A}u',\qquad
{\cal A}(\alpha u)=\alert{\bar\alpha} {\cal A}u
\]
for all $u,u'\in U$ and $\alpha
\in\mathbb C$.
\bigskip



We write
\begin{itemize}
  \item ${\cal A}:
      U\alert{\longrightarrow} V$
      if ${\cal A}$ is a
      \textcolor{blue}{linear
mapping}, and
  \item ${\cal A}:
      U\alert{\dashrightarrow} V$
      if ${\cal A}$ is a
      \textcolor{blue}{semilinear
      mapping}.
\end{itemize}
\end{frame}


\section{Main result}
\begin{frame}[t]
\frametitle{Cycles of linear and
semilinear mappings}


We give a canonical form of matrices of
a \alert{cycle of linear and semilinear
mappings}
\[
\xymatrix{
{V_1}
\ar@{-}@/_1.5pc/[rrrr]^{\mathcal A_t}
\ar@{-}[r]^{\mathcal A_1}&
V_2\ar@{-}[r]^{\mathcal A_2\ } &
{\ \dots\ }&{V_{t-1}}
\ar@{-}[l]_{\mathcal A_{t-2}}
\ar@{-}[r]^{\ \mathcal A_{t-1}}&{V_t}}
\]
in which \textcolor{blue}{each line is}
\begin{itemize}
  \item a full arrow
      \alert{$\longrightarrow$},
      \alert{$\longleftarrow$}, or
  \item a dashed arrow
      \alert{$\dashrightarrow$},
      \alert{$\dashleftarrow$}.
\end{itemize}


\end{frame}


\subsection{Bibliography}
\begin{frame}[t]
\frametitle{My talk is based on:}

\begin{itemize}

  \item \alert{T. Klimchuk,
  D. Kovalenko, T. Rybalkina,
  V.V. Sergeichuk}, {\it Tame systems
  of linear and semilinear mappings}),
   Contemp. Math. 658 (2016) 103-114.
\bigskip
\bigskip


  \item \alert{D. Duarte de
      Oliveira, V. Futorny, T.
      Klimchuk, D. Kovalenko, V.V.
      Sergeichuk}, {\it Cycles of
      linear and semilinear
      mappings}), Linear Algebra
      Appl. 438 (2013) 3442-3453.
\bigskip
\bigskip

  \item \alert{D. Duarte de
      Oliveira, R.A. Horn, T.
      Klimchuk, V.V. Sergeichuk},
      {\it Remarks on the
      classification of a pair of
      commuting semilinear
      operators}, Linear Algebra
      Appl. 436 (2012) 3362-3372.



\end{itemize}



\end{frame}
\subsection{Preliminaries}
\begin{frame}[t]
\frametitle{Empty matrices}

\begin{itemize}
  \item $\forall n=0,1,2,\dots$
      $\exists !$ matrices of sizes
      \alert{$0\times n$} and
      \alert{$n\times 0$},

      which correspond to linear
  mappings
  \textcolor{blue}{$\mathbb C^n\to
  0$} and
  \textcolor{blue}{$0\to\mathbb
      C^n$}.
\medskip

  \item They are denoted by
      \alert{$0_{0n}$} and
      \alert{$0_{n0}$} and are
      considered as zero matrices
\medskip

  \item For every $p\times q$
      matrix $M_{pq}$:
\begin{align*}
\alert{M_{pq}\oplus 0_{n0}}&=\begin{bmatrix}
  M_{pq} & 0 \\
  0 &0_{n0}
\end{bmatrix}=\begin{bmatrix}
  M_{pq}& 0_{p0} \\
  0_{nq}& 0_{n0}
\end{bmatrix}=\alert{\begin{bmatrix}
M_{pq} \\ 0_{nq}
\end{bmatrix}}
\\[5mm]
\alert{M_{pq}\oplus 0_{0n}}&=\begin{bmatrix}
  M_{pq} & 0 \\
  0 & 0_{0n}
\end{bmatrix}=\begin{bmatrix}
  M_{pq}& 0_{pn} \\
  0_{0q}& 0_{0n}
\end{bmatrix}=\alert{\begin{bmatrix}
   M_{pq} & 0_{pn}
\end{bmatrix}}
\end{align*}

\end{itemize}
  \end{frame}

\begin{frame}[t]
\frametitle{\alert{Soooooo baaad}
without empty matrices}

 $\forall A$ $\exists$
      nonsingular $R,S$:
      $RAS=\begin{bmatrix}
              I & 0 \\
              0 & 0 \\
            \end{bmatrix}$

\textcolor{blue}{Observation}
\begin{itemize}
  \item Each matrix is equivalent
      to a direct sum of
      indecomposable matrices of
      the form \[ \alert{[1],\ [1\,
      0],\ [1\,0\,0],\ \dots,\
\begin{bmatrix}
  1 \\0 \\
\end{bmatrix},\ \begin{bmatrix}
  1 \\0 \\0
\end{bmatrix},\ \dots}
\]

  \item This direct sum \alert{is
      not uniquely determined}:
\begin{align*}
\begin{bmatrix}
  1&0&0 \\0&0&0\\
\end{bmatrix}&=\left[\begin{array}{cc|c}
1&0&0 \\\hline 0&0&0\end{array}\right]=
\alert{[1\,0]\oplus [0]}
\\
&=\left[\begin{array}{c|cc}
1&0&0 \\\hline 0&0&0\end{array}\right]=
\alert{[1]\oplus [0\, 0]}
\end{align*}
\end{itemize}
  \end{frame}

\begin{frame}[t]
\frametitle{ \alert{Gooood}
with empty
matrices} \vspace{-1cm}
\begin{align*}
\begin{bmatrix}
  1&0&0 \\0&0&0\\
\end{bmatrix}&=
[1\,0]\oplus [0]
=
\alert{[1]\oplus 0_{01} \oplus
0_{01}\oplus 0_{10}}
\\
&=
[1]\oplus [0\, 0]=
\alert{[1]\oplus 0_{01} \oplus
0_{01}\oplus 0_{10}}
\end{align*}
\bigskip

\textcolor{blue}{\large A la Jordan
Theorem}
\begin{itemize}
  \item \textcolor{blue}{\it Each
      matrix is equivalent to a
      direct sum of indecomposable
      matrices of the form}
      \alert{$[1]$, $0_{01}$,
      $0_{10}$}.

  \item \textcolor{blue}{\it This
      direct sum is}
      \alert{uniquely
      determined}\textcolor{blue}{\it,
      up to permutation of
      summands.}
\end{itemize}
  \end{frame}



\end{document}

